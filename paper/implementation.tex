\documentclass{article}

\begin{document}
\chapter{Implementation}

This chapter describes PLSI's implementation.  I'll start with a bit of the
philophisy behind producing a portable flow.

\section{PLSI Makefiles}

Nobody knows anything about \texttt{make}, and I'm not entirely sure why.  It
seems like lots of people either go and re-invent \texttt{make} just because
they weren't aware it's possible to implement something in \texttt{make}, or
just don't understand the concept of dependencies.  I thought I'd spend a
section here describing:

\begin{itemize}
\item What ``\$@'' does
\item Why you should write Makefiles that check error messages and have
dependencies.
\item How I went about writing Makefiles that actually work.
\end{itemize}

\section{pcad}

One of the things I wanted to do as part of the big rewrite was to
essentially start writing my own CAD tools.  The idea here is that the
commercial CAD tools are both awesome (ie, DC is magic about QoR stuff) and
horrible (they don't do simple things like allowing for
technology-independent macro mapping).  This section will describe how pcad
is implemented, with the first 

\subsection{Intermediate Representation}

I spent a lot of time when writing libflo figuring out how to produce
type-safe IRs that could be manipulated by multiple tools, without resorting
to the sort of hacks that FIRRTL needs to.  Here I'll describe how I write
IRs using C++ templates, why the other versions are garbage, and how the
match stuff works for me.

\subsection{Macro Compiler}

One big chunk of code is the pcad macro compiler, which I think is something
other people should really be using.  As far as I know there really isn't
anything else out there like it: Synopsys doesn't do this, and it blows
Yunsup's stuff ou of the water.

\subsection{Technology Mapping}

pcad enforces the split between technology-independent models and
technology-dependent models/implementations.  This is one of the mechanisms
by which PLSI attempts to maintain portability: there must be a
technology-independent implementation of every macro, and that must simulate
correctly.

\section{Floorplanning}

Since I'm going to throw all this away at the end of the thesis, I'm not
going to bother re-implementing the floorplanning stuff in pcad.

\section{Verification}

PLSI integrates verification very tightly into the design workflow.  It goes
so far as to use some very complicated \texttt{make} mechanisms (ones that are
explicitly designed to be hard enough that circuits people can't disable
them) to enforce that you actually run the tests, as I've found that unless
you force people to test their stuff they'll just lie about it working.

\end{document}
