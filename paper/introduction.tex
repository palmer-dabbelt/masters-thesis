\documentclass{article}

\begin{document}
\chapter{Introduction}

Computer architecture research has historically been handicapped by a lack of
high quality baselines.  Thanks to the advent of RISC-V and Rocket Chip, there
are now high quality baseline implementations of multiple families of
microarchitectures available for free and with permissive licenses, along with
silicon-proven results for some benchmarks.  Unfortunately there is no existing
VLSI flow that allow architecture researchers to reproduce these results and to
obtain numbers for their proposed ideas.

PLSI is designed to be this flow: specifically PLSI is designed to allow
computer architecture researchers to quickly get VLSI numbers, allowing them to
do RTL based research.  To achieve this, PLSI uses standard commercial VLSI
tools and flows whenever possible -- for this thesis only the Synopsys-based
flow is demonstrated, but there are experimental flows for both Cadence and
open-source tools.  In addition to being portable between multiple tools
vendors, PLSI is designed to be portable to multiple designs (to enable
researchers to build their own cores) and to multiple processes (since
researchers tend to have access to an odd set of technologies).

The specific contributions of PLSI are:

\begin{itemize}
\item A macro synthesis tool used to map user's memories to various flavors of
ASIC SRAMs.
\item A modular, technology-independent floorplanning framework.
\item Criteria for ensuring the commercial CAD tools produce reasonable results.
\item Usability improvements from a good top-level Makefile that allows the
entire flow to be automated and reproducible.
\end{itemize}

This thesis presents results for three Chisel-based cores (Rocket, BOOM, and
Hwacha) mapped to Synopsys's educational 32nm technology via the Synopsys
tools.

\section{Sponsors and Such}

\TODO{I never know what to put here.}

\end{document}
