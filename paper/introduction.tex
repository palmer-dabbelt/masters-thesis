\documentclass{article}

\begin{document}
\chapter{Introduction}

With the advent of Chisel and Rocket Chip we can do a good job producing many
RTL-level designs.  Unfortunately we're still a bit stuck in the past when it
comes to VLSI flows: even at Berkeley where we build lots of chips, it still
takes a year or more to re-spin a new design, even one that's very similar to
the old one.  These turnaround times make it impossible to do computer
architecture research using the VLSI flows used to build chips, which is a
major pain on both ends. 

\section{History}

After becoming frustrated working on multiple failed tapeouts I decided to
build a new VLSI flow that's explicitly designed to be portable.  The
original goal was to produce a VLSI flow that was sufficient to build the
sort of test chips we build at Berkeley, but also malleable enough to be used
for computer architecture research.

I set about producing a VLSI flow that met my needs.  Specifically my goal
when designing chips was to change from a mentality of building one specific
design to one of building many designs at once.

The original version of this VLSI flow was used in a single tapeout on TSMC's
16nm FinFET technology, but it was a big mess.  After this tapeout I decided
there were too many political problems preventing the original goal from
being possible, so I instead decided to retool this project as a tool for
computer architecture research.

\section{Related Work}

\TODO{I don't know of any related work...}

I think the differences are:

\begin{itemize}
\item This uses real technologies, real CAD tools, and real RTL designs.
\item This flow is pretty close to producing a chip, just not one that's
suitable for circuits research projects.
\item This flow is designed to support multiple technologies, tools, and
designs.
\item This flow is open source.
\end{itemize}

I think what it boils down to is that the people doing circuits research
don't have the right goals to produce a flow like this, since they're focused
on building one chip at a time.  The people doing computer architecture
research usually don't even have RTL, and if they have VLSI results at all
they're usually a big mess and not something that's possible to be released
or used by others.

\section{Sponsors and Such}

\TODO{I never know what to put here.}

\end{document}
