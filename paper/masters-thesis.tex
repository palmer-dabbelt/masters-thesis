\documentclass[masters2]{ucbthesis}
\usepackage{graphicx}
\usepackage{multicol}
\usepackage{hyperref}

\author{Palmer Dabbelt}
\title{PLSI: A Portable VLSI Flow}
\degree{Master of Science}
\field{Computer Science}
\degreeyear{2017}
\degreesemester{Spring}
\chair{Krste Asanovic}
\othermembers{Jonathan Bachrach}
\numberofmembers{2}
\campus{Berkeley}

\newcommand{\TODO}[1]{#1}

\begin{document}
\thispagestyle{empty}
\large
\begin{center}
\textbf{PLSI: A Portable VLSI Flow}\\
By Palmer Dabbelt\\
\textbf{Research Project}\\
\end{center}

\normalsize
Submitted to the Department of Electrical Engineering and Computer Sciences,
University of California at Berkeley, in partial satisfaction of the
requirements for the degree of Master of Science, Plan II.

Approval for the Report and Comprehensive Examination:

\begin{center}
Committee:\\
\includegraphics[width=2in]{figures/krste.png}
\rule{5in}{1pt}\\
Professor Krste Asanovic\\
Research Advisor\\
\vspace{0.5in}\rule{5in}{1pt}\\
(Date)\\
\vspace{0.5in}\rule{5in}{1pt}\\
Professor Jonathan Bachrach\\
Second Reader\\
\vspace{0.5in}\rule{5in}{1pt}\\
(Date)\\
\end{center}

\copyrightpage

\begin{frontmatter}

\tableofcontents
\clearpage
\listoffigures

\begin{abstract}
This report presents PLSI, a portable VLSI flow designed to enable RTL-based
computer architecture research.  The interesting part of PLSI are the tools
that implement the various rules and the interchange formats that are passed
between the various tools.  The fundamental driving design decision behind PLSI
is that computers are better than performing repetitive, arithmetic-laden tasks
that humans are.  When implementing PLSI I took my experience from working with
a handful of tapeout teams.  This report present implementations of Rocket,
Hwacha, and BOOM on the Synopses 32nm Educational Technology.
\end{abstract}

\begin{acknowledgements}
I would like thank my fellow graduate students: Colin Schmidt and Ben Keller
for helping work on our chips, Andrew Waterman and Yunsup Lee for giving me lot
of advice during my studies, Chris Celio for providing BOOM, as well as the
Berkeley Architecture Research Group.  Without them, as well as the numerous
other contributors to both Rocket Chip and the general RISC-V ecosystem, I
would never have had all the infrastructure I needed to even start this thesis.

I would also like to thank my advisors Krste Asanovic and Jonathan Barchrach
for providing advice throughout my graduate studies, as well as John
Kubiatowicz who was my advisor for my first two years at Berkeley.  I would
also like to thank Dave Patterson, who helped while finishing my thesis.

Finally, I would like to thank Lulu Li and my parents for always being around
when I need them.
\end{acknowledgements}

\end{frontmatter}

\pagestyle{headings}

\documentclass{article}

\begin{document}
\chapter{Introduction}

With the advent of Chisel and Rocket Chip we can do a good job producing many
RTL-level designs.  Unfortunately we're still a bit stuck in the past when it
comes to VLSI flows: even at Berkeley where we build lots of chips, it still
takes a year or more to re-spin a new design, even one that's very similar to
the old one.  These turnaround times make it impossible to do computer
architecture research using the VLSI flows used to build chips, which is a
major pain on both ends. 

\section{History}

After becoming frustrated working on multiple failed tapeouts I decided to
build a new VLSI flow that's explicitly designed to be portable.  The
original goal was to produce a VLSI flow that was sufficient to build the
sort of test chips we build at Berkeley, but also malleable enough to be used
for computer architecture research.

I set about producing a VLSI flow that met my needs.  Specifically my goal
when designing chips was to change from a mentality of building one specific
design to one of building many designs at once.

The original version of this VLSI flow was used in a single tapeout on TSMC's
16nm FinFET technology, but it was a big mess.  After this tapeout I decided
there were too many political problems preventing the original goal from
being possible, so I instead decided to retool this project as a tool for
computer architecture research.

\section{Related Work}

\TODO{I don't know of any related work...}

I think the differences are:

\begin{itemize}
\item This uses real technologies, real CAD tools, and real RTL designs.
\item This flow is pretty close to producing a chip, just not one that's
suitable for circuits research projects.
\item This flow is designed to support multiple technologies, tools, and
designs.
\item This flow is open source.
\end{itemize}

I think what it boils down to is that the people doing circuits research
don't have the right goals to produce a flow like this, since they're focused
on building one chip at a time.  The people doing computer architecture
research usually don't even have RTL, and if they have VLSI results at all
they're usually a big mess and not something that's possible to be released
or used by others.

\section{Sponsors and Such}

\TODO{I never know what to put here.}

\end{document}

\documentclass{article}

\begin{document}
\chapter{Implementation}

This chapter describes PLSI's implementation.  I'll start with a bit of the
philophisy behind producing a portable flow.

\section{PLSI Makefiles}

Nobody knows anything about \texttt{make}, and I'm not entirely sure why.  It
seems like lots of people either go and re-invent \texttt{make} just because
they weren't aware it's possible to implement something in \texttt{make}, or
just don't understand the concept of dependencies.  I thought I'd spend a
section here describing:

\begin{itemize}
\item What ``\$@'' does
\item Why you should write Makefiles that check error messages and have
dependencies.
\item How I went about writing Makefiles that actually work.
\end{itemize}

\section{pcad}

One of the things I wanted to do as part of the big rewrite was to
essentially start writing my own CAD tools.  The idea here is that the
commercial CAD tools are both awesome (ie, DC is magic about QoR stuff) and
horrible (they don't do simple things like allowing for
technology-independent macro mapping).  This section will describe how pcad
is implemented, with the first 

\subsection{Intermediate Representation}

I spent a lot of time when writing libflo figuring out how to produce
type-safe IRs that could be manipulated by multiple tools, without resorting
to the sort of hacks that FIRRTL needs to.  Here I'll describe how I write
IRs using C++ templates, why the other versions are garbage, and how the
match stuff works for me.

\subsection{Macro Compiler}

One big chunk of code is the pcad macro compiler, which I think is something
other people should really be using.  As far as I know there really isn't
anything else out there like it: Synopsys doesn't do this, and it blows
Yunsup's stuff ou of the water.

\subsection{Technology Mapping}

pcad enforces the split between technology-independent models and
technology-dependent models/implementations.  This is one of the mechanisms
by which PLSI attempts to maintain portability: there must be a
technology-independent implementation of every macro, and that must simulate
correctly.

\section{Floorplanning}

Since I'm going to throw all this away at the end of the thesis, I'm not
going to bother re-implementing the floorplanning stuff in pcad.

\section{Verification}

PLSI integrates verification very tightly into the design workflow.  It goes
so far as to use some very complicated \texttt{make} mechanisms (ones that are
explicitly designed to be hard enough that circuits people can't disable
them) to enforce that you actually run the tests, as I've found that unless
you force people to test their stuff they'll just lie about it working.

\end{document}

\documentclass{article}
\usepackage{multicol}

\begin{document}
\chapter{Results}

PLSI has been implemented and works for multiple Rocket Chip based designs.
This section presents the results for Rocket, BOOM, and Hwacha on Synopsys'
32nm Educational PDK produced via the Synopsys tools.  In addition to
presenting the results, there is a discussion of the validity of the results.

These results were all run using the latest version of Rocket Chip that
supported the various configurations when this thesis was being written.  The
exact versions of Rocket Chip differ between the various targets because all
the forks aren't up to date at all times, but they're all tagged as git
submodules in the thesis repository.  Newer versions of Rocket Chip have
removed support for the L2 cache, which is why the Hwacha configuration (which
is based on an older version of Rocket Chip) is the only configuration that has
an L2 cache attached.

\subsection{Rocket}

\begin{figure}
  \begin{verbatim}
{
  "clocks": [
    {
      "name": "clock",
      "period": "1250ps",
      "par derating": "250ps"
    }
  ],
  "scenarios": [
    {
      "corner": {
        "nmos": "typical",
        "pmos": "typical",
        "temperature": "25 C"
      },
      "supplies": {
        "VDD": "1.05 V",
        "GND": "0 V"
      }
    }
  ]
}
\end{verbatim}
  \caption{PLSI CAD Configuration for DefaultConfig}
  \label{res:rocket-config}
\end{figure}

\begin{figure}
\tiny
\begin{verbatim}
class RocketTilePlacer:
  def __init__(self, config):
    self.top = None
    self.l1dd = []
    self.l1dm = []
    self.l1id = []
    self.l1im = []

  def insert(self, macro):
    if macro.matches(config.rtl_top):
      self.top = TopMacro(macro.name, macro.width, macro.height)
    # Version Break
    elif macro.matches("coreplex/RocketTile/DCache/data"):
      self.l1dd.append(macro)
    elif macro.matches("coreplex/RocketTile/DCache/MetadataArray"):
      self.l1dm.append(macro)
    elif macro.matches("coreplex/RocketTile/icache/icache"):
      self.l1id.append(macro)
    elif macro.matches("coreplex/RocketTile/icache/icache/tag_array/tag_array"):
      self.l1im.append(macro)
    # Version Break
    elif macro.matches("coreplex/rocketTiles/dcache/data"):
      self.l1dd.append(macro)
    elif macro.matches("coreplex/rocketTiles/dcache/MetadataArray"):
      self.l1dm.append(macro)
    elif macro.matches("coreplex/rocketTiles/frontend/icache"):
      self.l1id.append(macro)
    elif macro.matches("coreplex/rocketTiles/frontend/icache/u"):
      self.l1id.append(macro)
    elif macro.matches("coreplex/rocketTiles/frontend/icache/tag_array/tag_array"):
      self.l1im.append(macro)
    else:
      print("%s cannot be matched" % macro.name)
      return False
    return True

  def list_constraints(self):
    l1dd = TopLeftPlacer   (self.top, self.top.tlf(), sorted(self.l1dd), True)
    l1dm = TopLeftPlacer   (self.top, l1dd.bl(),      sorted(self.l1dm), False)
    l1id = BottomLeftPlacer(self.top, self.top.blf(), sorted(self.l1id), True)
    l1im = BottomLeftPlacer(self.top, l1id.tl(),      sorted(self.l1im), False)
    return l1dd.place() + l1dm.place() + l1id.place() + l1im.place()
\end{verbatim}
  \caption{PLSI Floorplan for DefaultConfig}
  \label{res:rocket-fppy}
\end{figure}

\begin{figure}
  \begin{center}
    \includegraphics[width=0.95\linewidth]{figures/icc-rocket.png}
  \end{center}
  \caption{ICC Floorplan for DefaultConfig}
  \label{res:rocket-icc}
\end{figure}

\begin{figure}
\begin{multicols}{2}
\begin{verbatim}
  Timing Path Group 'clock'
  -----------------------------------
  Levels of Logic:              34.00
  Critical Path Length:          1.52
  Critical Path Slack:          -0.16
  Critical Path Clk Period:      1.35
  Total Negative Slack:        -18.73
  No. of Violating Paths:      988.00
  Worst Hold Violation:          0.00
  Total Hold Violation:          0.00
  No. of Hold Violations:        0.00
  -----------------------------------
\end{verbatim}

\begin{verbatim}
  Cell Count
  -----------------------------------
  Hierarchical Cell Count:        609
  Hierarchical Port Count:      18116
  Leaf Cell Count:             115987
  Buf/Inv Cell Count:           16638
  Buf Cell Count:                7235
  Inv Cell Count:                9403
  CT Buf/Inv Cell Count:          689
  Combinational Cell Count:     94266
  Sequential Cell Count:        21721
  Macro Count:                     60
  -----------------------------------
\end{verbatim}

\begin{verbatim}
  Area
  -----------------------------------
  Combinational Area:   258124.149879
  Noncombinational Area:
                        144030.016922
  Buf/Inv Area:          48915.350153
  Total Buffer Area:         31126.03
  Total Inverter Area:       17789.32
  Macro/Black Box Area:
                       1009378.968750
  Net Area:             381408.759989
  Net XLength        :     2435133.75
  Net YLength        :     2365051.00
  -----------------------------------
  Cell Area:           1411533.135552
  Design Area:         1792941.895541
  Net Length        :      4800185.00


  Design Rules
  -----------------------------------
  Total Number of Nets:        125786
  Nets With Violations:           316
  Max Trans Violations:            14
  Max Cap Violations:             308
\end{verbatim}
\end{multicols}
  \caption{ICC QoR Report for DefaultConfig}
  \label{res:rocket-qor}
\end{figure}

\clearpage
\subsection{BOOM}

\begin{figure}
  \begin{verbatim}
{
  "clocks": [
    {
      "name": "clock",
      "period": "1600ps",
      "par derating": "400ps"
    }
  ],
  "scenerios": [
    {
      "corner": {
        "nmos": "typical",
        "pmos": "typical",
        "temperature": "25 C"
      },
      "supplies": {
        "VDD": "1.05 V",
        "GND": "0 V"
      }
    }
  ]
}
\end{verbatim}
  \caption{PLSI CAD Configuration for SmallBOOMConfig}
  \label{res:boom-config}
\end{figure}

\begin{figure}
\tiny
\begin{verbatim}
class BoomTilePlacer(RocketTilePlacer):
  def __init__(self, config):
    self.top = None
    self.l1dd = []
    self.l1dm = []
    self.l1id = []
    self.l1im = []
    self.l1ht = []
    self.l1pt = []
    self.l1ei = []

  def insert(self, macro):
    if macro.matches(config.rtl_top):
      self.top = TopMacro(macro.name, macro.width, macro.height)
    # Version Break
    elif macro.matches("coreplex/BOOMTile/DCache/data"):
      self.l1dd.append(macro)
    elif macro.matches("coreplex/BOOMTile/DCache/MetadataArray"):
      self.l1dm.append(macro)
    elif macro.matches("coreplex/BOOMTile/HellaCache/MetadataArray"):
      self.l1dm.append(macro)
    elif macro.matches("coreplex/BOOMTile/HellaCache/meta"):
      self.l1dm.append(macro)
    elif macro.matches("coreplex/BOOMTile/icache/icache"):
      self.l1id.append(macro)
    elif macro.matches("coreplex/BOOMTile/icache/icache/u"):
      self.l1id.append(macro)
    elif macro.matches("coreplex/BOOMTile/HellaCache/data"):
      self.l1dd.append(macro)
    elif macro.matches("coreplex/BOOMTile/icache/icache/tag_array/tag_array"):
      self.l1im.append(macro)
    elif macro.matches("coreplex/BOOMTile/core/bpd_stage/br_predictor/counters/p_table/p_table/p_table"):
      self.l1pt.append(macro)
    elif macro.matches("coreplex/BOOMTile/core/bpd_stage/br_predictor/counters/h_table/h_table/h_table"):
      self.l1ht.append(macro)
    elif macro.matches("coreplex/BOOMTile/core/bpd_stage/br_predictor/brob/entries_info/entries_info"):
      self.l1ei.append(macro)
    else:
      print("%s cannot be matched" % macro.name)
      return False
    return True

  def list_constraints(self):
    l1dd = TopLeftPlacer   (self.top, self.top.tlf(), sorted(self.l1dd), True)
    l1dm = TopLeftPlacer   (self.top, l1dd.bl(),      sorted(self.l1dm), False)
    l1ht = BottomLeftPlacer(self.top, self.top.blf(), sorted(self.l1ht), True)
    l1pt = BottomLeftPlacer(self.top, l1ht.tl(),      sorted(self.l1pt), False)
    l1id = BottomLeftPlacer(self.top, l1pt.tl(),      sorted(self.l1id), False)
    l1im = BottomLeftPlacer(self.top, l1id.tl(),      sorted(self.l1im), False)
    l1ei = BottomLeftPlacer(self.top, l1im.tl(),      sorted(self.l1ei), False)
    return l1dd.place() + l1dm.place()  + l1ht.place() + l1pt.place() + l1id.place() + l1im.place() + l1ei.place()
\end{verbatim}
  \caption{PLSI Floorplan for SmallBOOMConfig}
  \label{res:boom-fppy}
\end{figure}

\begin{figure}
  \begin{center}
    \includegraphics[width=0.95\linewidth]{figures/icc-boom.png}
  \end{center}
  \caption{ICC Floorplan for SmallBOOMConfig}
  \label{res:boom-icc}
\end{figure}

\begin{figure}
\begin{multicols}{2}
\begin{verbatim}
  Timing Path Group 'clock'
  -----------------------------------
  Levels of Logic:              53.00
  Critical Path Length:          3.56
  Critical Path Slack:          -1.56
  Critical Path Clk Period:      2.00
  Total Negative Slack:      -1282.67
  No. of Violating Paths:    10636.00
  Worst Hold Violation:          0.00
  Total Hold Violation:          0.00
  No. of Hold Violations:        0.00
  -----------------------------------
\end{verbatim}

\begin{verbatim}
  Cell Count
  -----------------------------------
  Hierarchical Cell Count:       1165
  Hierarchical Port Count:      76737
  Leaf Cell Count:             226528
  Buf/Inv Cell Count:           39474
  Buf Cell Count:               20757
  Inv Cell Count:               18717
  CT Buf/Inv Cell Count:         2089
  Combinational Cell Count:    187351
  Sequential Cell Count:        39177
  Macro Count:                    186
  -----------------------------------
\end{verbatim}

\begin{verbatim}
  Area
  -----------------------------------
  Combinational Area:   548249.859481
  Noncombinational Area:
                        260366.723512
  Buf/Inv Area:         141541.690599
  Total Buffer Area:         94002.53
  Total Inverter Area:       47539.16
  Macro/Black Box Area:
                       3912562.714844
  Net Area:            1230432.321193
  Net XLength        :     7458008.00
  Net YLength        :     7564863.50
  -----------------------------------
  Cell Area:           4721179.297837
  Design Area:         5951611.619030
  Net Length        :     15022872.00
\end{verbatim}
\end{multicols}
  \caption{ICC QoR Report for SmallBOOMConfig}
  \label{res:boom-qor}
\end{figure}

\clearpage
\subsection{Hwacha (with L2)}

\begin{figure}
  \begin{verbatim}
{
  "clocks": [
    {
      "name": "clock",
      "period": "2500ps",
      "par derating": "2500ps"
    }
  ],
  "scenerios": [
    {
      "corner": {
        "nmos": "typical",
        "pmos": "typical",
        "temperature": "25 C"
      },
      "supplies": {
        "VDD": "1.05 V",
        "GND": "0 V"
      }
    }
  ]
}
\end{verbatim}
  \caption{PLSI CAD Configuration for EOS24Config}
  \label{res:hwacha-config}
\end{figure}

\begin{figure}
\tiny
\begin{verbatim}
class HwachaTilePlacer(RocketTilePlacer):
  def __init__(self, config):
    self.top = None
    self.l1dd = []
    self.l1dm = []
    self.l1id = []
    self.l1im = []
    self.hid = []
    self.him = []
    self.hrf = []
    self.l2d = []
    self.l2m = []

  def insert(self, macro):
    if macro.matches(config.rtl_top):
      self.top = TopMacro(macro.name, macro.width, macro.height)
    elif macro.matches("DefaultCoreplex/tiles/HellaCache/data"):
      self.l1dd.append(macro)
    elif macro.matches("DefaultCoreplex/tiles/HellaCache/meta"):
      self.l1dm.append(macro)
    elif macro.matches("DefaultCoreplex/tiles/icache/icache"):
      self.l1id.append(macro)
    elif macro.matches("DefaultCoreplex/tiles/icache/icache/u"):
      self.l1id.append(macro)
    elif macro.matches("DefaultCoreplex/tiles/icache/icache/tag_array/tag_array"):
      self.l1im.append(macro)
    elif macro.matches("DefaultCoreplex/tiles/Hwacha/icache/icache"):
      self.hid.append(macro)
    elif macro.matches("DefaultCoreplex/tiles/Hwacha/icache/icache/u"):
      self.hid.append(macro)
    elif macro.matches("DefaultCoreplex/tiles/Hwacha/icache/icache/tag_array/tag_array"):
      self.him.append(macro)
    elif macro.matches("DefaultCoreplex/tiles/Hwacha/vus/vxuInst/laneInst/bankInst/rfInst/HwSRAMRF/HwSRAMR"):
      self.hrf.append(macro)
    elif macro.matches("DefaultCoreplex/tiles/Hwacha/vus/vxuInst/laneInst/bankInst/rfInst/HwSRAMRF/HwSRAMRF"):
      self.hrf.append(macro)
    elif macro.matches("DefaultCoreplex/L2HellaCacheBank/data_array/array"):
      self.l2d.append(macro)
    elif macro.matches("DefaultCoreplex/L2HellaCacheBank/meta/meta"):
      self.l2m.append(macro)
    else:
      print("%s cannot be matched" % macro.name)
      return False
    return True

  def list_constraints(self):
    l1dd = TopLeftPlacer   (self.top, self.top.tlf(), sorted(self.l1dd), True,  0.2)
    l1dm = TopLeftPlacer   (self.top, l1dd.bl(),      sorted(self.l1dm), False, 0.2)
    l1id = TopLeftPlacer   (self.top, l1dm.bl(),      sorted(self.l1id), False, 0.2)
    l1im = TopLeftPlacer   (self.top, l1id.bl(),      sorted(self.l1im), False, 0.2)
    hrf  = BottomLeftPlacer(self.top, self.top.blf(), sorted(self.hrf),  True,  0.2)
    hid  = BottomLeftPlacer(self.top, hrf.tl(),       sorted(self.hid),  False, 0.2)
    him  = BottomLeftPlacer(self.top, hid.tl(),       sorted(self.him),  False, 0.2)
    l2d  = TopRightPlacer  (self.top, self.top.trf(), sorted(self.l2d),  True,  0.8)
    l2m  = TopRightPlacer  (self.top, l2d.br(),       sorted(self.l2m),  False, 0.8)
    return l1dd.place() + l1dm.place() + l1id.place() + l1im.place() + hrf.place()
           + hid.place() + him.place() + l2d.place() + l2m.place()
\end{verbatim}
  \caption{PLSI Floorplan for EOS24Config}
  \label{res:hwacha-fppy}
\end{figure}

\begin{figure}
  \begin{center}
    \includegraphics[width=0.95\linewidth]{figures/icc-hwacha.png}
  \end{center}
  \caption{ICC Floorplan for EOS24Config}
  \label{res:hwacha-icc}
\end{figure}

\begin{figure}
\begin{multicols}{2}
\begin{verbatim}
  Timing Path Group 'clock'
  -----------------------------------
  Levels of Logic:              29.00
  Critical Path Length:          5.63
  Critical Path Slack:          -0.73
  Critical Path Clk Period:      5.00
  Total Negative Slack:         -2.61
  No. of Violating Paths:       27.00
  Worst Hold Violation:          0.00
  Total Hold Violation:          0.00
  No. of Hold Violations:        0.00
  -----------------------------------
\end{verbatim}
\begin{verbatim}
  Cell Count
  -----------------------------------
  Hierarchical Cell Count:       3517
  Hierarchical Port Count:     145207
  Leaf Cell Count:             786039
  Buf/Inv Cell Count:          144033
  Buf Cell Count:               65811
  Inv Cell Count:               78222
  CT Buf/Inv Cell Count:         3501
  Combinational Cell Count:    686589
  Sequential Cell Count:        99450
  Macro Count:                    608
  -----------------------------------
\end{verbatim}
\begin{verbatim}
  Area
  -----------------------------------
  Combinational Area:  1875823.402210
  Noncombinational Area:
                        659483.367816
  Buf/Inv Area:         463893.370644
  Total Buffer Area:        283467.90
  Total Inverter Area:      180425.47
  Macro/Black Box Area:
                      10255909.371094
  Net Area:            3823554.904260
  Net XLength        :    28303776.00
  Net YLength        :    25156786.00
  -----------------------------------
  Cell Area:          12791216.141120
  Design Area:        16614771.045380
  Net Length        :     53460560.00


  Design Rules
  -----------------------------------
  Total Number of Nets:        850635
  Nets With Violations:         13080
  Max Trans Violations:          1133
  Max Cap Violations:           12164
\end{verbatim}
\end{multicols}
  \caption{ICC QoR Report for EOS24Config}
  \label{res:hwacha-qor}
\end{figure}

\section{Discussion}

The results in this section are meant to be suitable for doing computer
architecture research, not for building an actual chip.  As such they skip the
signoff sections of the flow that, from my experience when building chips,
don't have meaningful effects on QoR but take a lot of time to get right.

The simplest configuration run for this thesis was the default Rocket
configuration, DefaultConfig.  This contains Rocket, 64-bit a 5 stage in-order
core, with 32 KiB split L1 caches and a floating-point unit.  This
configuration and floorplan gets very good performance (1.5GHz) on commercial
28nm technologies, but it appears that the 32nm Synopsys EDK is significantly
slower.

The PLSI configuration used to generate DefaultConfig is shown in
Figure~\ref{res:rocket-config} and is pretty boring: there's a little bit of
derating for PAR, but that's just about as good as I'm able to do for anything.
The post-PAR results show that it's close to meeting timing, and that there
aren't too many DRC errors so the results are probably believable.

The Hwacha results are, however, a whole different story.  As you can see from
the PLSI configuration file, the physical design here gets very bad QoR --
there's a 50\% clock rate difference between the post-PAR and post-synthesis
results.  While some of this can be attributed to Rocket Chip's memories not
mapping well to the 32nm EDK's SRAMs (which have to write masks), a significant
part of this is probably due to the poor floorplan quality.  From looking at
Figure~\ref{res:hwacha-icc}, you can see how the floorplan generated from
Figure~\ref{res:hwacha-fppy} is quite bad: you can see how the L2 is too wide
and too short, which results in there being no space for the actual core logic.

\end{document}

\documentclass{article}

\begin{document}
\chapter{Conclusion}

\section{Future Work}

\subsection{Retiming}

\subsection{Estimating Clock Speeds}

Hopefully the conclusion is ``it's really not that hard to run VLSI results for
an ISCA paper, just use my stuff''.

\end{document}


\end{document}
