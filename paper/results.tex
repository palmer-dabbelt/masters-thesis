\documentclass{article}

\begin{document}
\chapter{Results}

A thesis needs some numbers (even though the hypothesis here is just that
``it's possible to produce this many numbers'').

\section{Benchmark Designs}

All the benchmark designs I'll be running will be based on Rocket Chip.  I'm
going to use three cores: Rocket, BOOM, and Hwacha; three technologies:
Synopsys 32nm EDK, TSMC 28HPM, and TSMC 16 FinFTE; and one CAD tool vendor:
Synopsys.  I looked at doing Cadence as well but it doesn't seem worth it
because the BWRC people won't give me any support (it just SEGVs when trying to
read some files and they won't install a new version).  If Bora makes too big
of a mess then I also have TSMC 40G+, some TSMC 65nm process, and a 180nm from
MOSIS/Oaklohoma State.

\section{Benchmark Software}

I'm going to kind of half-ass this here because I just want this to be done:
I'm just going to run the stuff in \texttt{riscv-benchmarks}.  That means DGEMM
and friends, nothing like SPEC.  I figure since the goal here is showing the
VLSI flow and not arguing for a particular architectural feature it'll be OK.
My out is going to be that SPEC takes a long time to run but with the magic
simulator stuff we'd be able to run it.

\section{Benchmark Methodology}

It's essentially just: run the benchmarks in RTL simulation to get cycle
counts, run the VLSI flow to get post-PNR RC numbers, and then run
PrimeTime-Power to get the energy results.

\section{Benchmark Results}

Here I would like to present scatter plots of the various designs on all the
technologies, answering those questions I proposed at the start of last
semester.  I think the questions will be something along the lines of:

\begin{itemize}
\item I need at least 100 GFlop/s, minimize energy.
\item I'm building 100k chips that cost \$10 each, maximize Dhrystone
performance.
\item How many convolutions/sec can I fit into 100mW?
\end{itemize}

The point is that these are the sort of questions you should be asking when
building a chip, just scaled down a bit so a single person project can produce
an interesting set of designs that will compete on them.

\end{document}
