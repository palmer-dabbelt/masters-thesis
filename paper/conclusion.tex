\documentclass{article}

\begin{document}
\chapter{Conclusion}

The goal of producing PLSI was to make it easy to generate VLSI results for
computer architecture research, and I think it's at least gotten part of the
way there.  Right now the biggest obstacle in using PLSI is that the
semi-automated floorplanning flow hasn't been pushed on enough to be solid.
Unfortunately I lost my access to real technologies right at the end of
producing this thesis so I couldn't get proper results, and Rocket Chip doesn't
map well to the 32nm Synopsys Educational PDK so what's in there won't apply
well to the other technologies I've seen.

\section{Future Work}

\subsection{Benchmarking and Power}

If you've been paying attention then you'll probably notice that there aren't
any results from actually simulating the designs that were pushed through the
tools.  This was mostly for time reasons (as far as I know PrimeTime Power
works and has been used by other people, but Donggyu fixed it after I'd ran out
of time to run results), but also because it appears that the 32nm Synopsys
EDK's power models wouldn't be good enough to get any reasonable information
from.  There's two problems here: the lack of write masks means many more SRAM
macros are emitted than is reasonable, and I get a lot of warnings from the
tools about SRAMs not having power modells.

\subsection{DRC and LVS Fixing}

I was hoping to have my designs LVS clean and set a low threshold for DRC
errors, but it looks like the 32nm Synopsys EDK has some problems that prevent
you from producing DRC clean results using it -- I specifically ran into a lot
of nets with redundant via errors that looked like some sort of tech file vs
DRC deck mismatch.  While I tried a few ways to fix this, I didn't have enough
time to get it fixed properly so just gave up.

\end{document}
