\documentclass{article}

\begin{document}
\chapter{Conclusion}

The goal of producing PLSI was to make it easy to generate VLSI results for
computer architecture research, and I think it's at least gotten part of the
way there.  Right now the biggest obstacle in using PLSI is that the
semi-automated floorplanning flow hasn't been pushed on enough to be solid.

\section{Future Work}

\subsection{A Useable Educational Technology}

I was limited to using the Synopsys 32nm EDK to evaluate PLSI for this thesis.
Unfortunately, optimizing designs for this technology results in design that
don't match any real technology I've seen because the cost of some key
structures doesn't match real technologies.

In order to actually be able to do RTL-based computer architecture research
without relying on the benevolence of other research groups, we need an
educational technology that produces results that match those produced by real
designs.  Since we're using this for computer architecture research instead of
VLSI or CAD research, this doesn't actually need things like a realistic DRC
deck.

While I think this is the most important future work for PLSI, I have no idea how
to go about solving this.

\subsection{Benchmarking and Power}

If you've been paying attention then you'll probably notice that there aren't
any results from actually simulating the designs that were pushed through the
tools.  This was mostly for time reasons (as far as I know PrimeTime Power
works and has been used by other people, but Donggyu fixed it after I'd ran out
of time to run results), but also because it appears that the 32nm Synopsys
EDK's power models wouldn't be good enough to get any reasonable information
from.  There's two problems here: the lack of write masks means many more SRAM
macros are emitted than is reasonable, and I get a lot of warnings from the
tools about SRAMs not having power models.

\subsection{DRC and LVS Fixing}

I was hoping to have my designs LVS clean and set a low threshold for DRC
errors, but it looks like the 32nm Synopsys EDK has some problems that prevent
you from producing DRC clean results using it -- I specifically ran into a lot
of nets with redundant via errors that looked like some sort of tech file vs
DRC deck mismatch.  While I tried a few ways to fix this, I didn't have enough
time to get it fixed properly so just gave up.

We've had these sorts of problems on real technologies as well, so I think this
actually warrants writing a proper tool.  This tool would read DRC decks,
technology files, and a set of DRC errors.  It would then modify the technology
file in order to elide these DRC errors.

\subsection{Clock Estimation}

Even when using the PLSI flow, users still have to manually write clock
constraints for every clock in their design.  This is an impediment when doing
RTL-based computer architecture research as you can't know if each of your
design points are producing reasonable QoR without performing some manual
optimization on each of them.

Based on my experience trying to get good QoR from our designs, it should be
possible to apply some simple heuristics to our RTL inputs in order to
determine the achievable clock period of a design to with 25\%, which is good
enough that the CAD tools won't blow up when asked to meet that timing
constraint.

\subsection{Trace Based SRAM Selection}

Given a well designed microprocessor, the most important constraint should be
how it maps to SRAMs.  Even when using PLSI this process still requires manual
intervention: while PLSI might make the selection of SRAMs automatic, the cost
function still requires an engineer to intuit how memories should be mapped
given their design constraints.  For PLSI this was easy: we just select the
fastest SRAM every time -- now that we're past the end of Dennard Scaling this
isn't the right thing to do.

The right way to fix this is to use simulation traces to drive the cost
function for SRAM selection, which will actually select the SRAM configuration
the user desires.  This is really just the same as trace-based synthesis, but
for SRAMs.

\subsection{An Accuracy vs Effort Curve for Physical Design}

While some might consider GDS the holy grail of accuracy when evaluating
designs, getting accurate results actually requires significant effort.  It
would be interesting to attempt to optimize extant designs to see how accurate
their results actually are, and to compare these results with those produced by
methodologies that requires less effort.

\end{document}
